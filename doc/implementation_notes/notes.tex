
\documentclass[a4paper]{article}
\usepackage[top=2cm, bottom=2cm, left=2.5cm, right=2.5cm]{geometry}


\usepackage[utf8]{inputenc}
\usepackage[english]{babel}
\usepackage{amsthm}
\usepackage{amsfonts}
\usepackage{amssymb}
\usepackage{hyperref}
\usepackage{biblatex}
\usepackage{listings}

\bibliography{notes}

\title{Bitcoin–Monero Cross-chain Atomic Swap Implementation notes\\
\normalsize v0.2 DRAFT}
\author{%
tecnovert\footnote{\texttt{tecnovert@tecnovert.net}}
}
\date{2020-09-10}

\begin{document}

\maketitle

\section{Introduction}
Notes on implementing the protocol described by h4sh3d in Bitcoin–Monero Cross-chain Atomic Swap\cite{h4sh3d:1}


\section{Discrete Logarithm Equality Across Groups}
A Discrete Logarithm Equality Across Groups (DLEAG) proof can prove a discrete logarithm is equal for two points on different curves.\\
We use the scheme from MRL-0010\cite{MRL:10}, slightly modified to use a borromean\cite{borromean} ring stack for a reduced proof size.
\\
The curves considered are secp256k1\cite{CerRes10} used by the Bitcoin\cite{nakamoto2008bitcoin} cryptocurrency and forks and ed25519\cite{ches-2011-24091} used by Monero\cite{van2013cryptonote}.\\
As the order of the generator point of the ed25519 curve $l$ is less than that of curve secp256k1 $o$ the shared secret must be less than $ l $.\\
To represent $l-1$ in base2, we must prove 252 bits.\\

\subsection{Definitions}
\begin{itemize}
    \item $G$   - The base point of curve secp256k1
    \item $\hat{G}$  - An alternate base point of curve secp256k1 in the same order as $G$, chosen as $G$ hashed to a point, proving it's discrete logarithm is unknown.
    \item $B$   - The base point of curve ed25519
    \item $\hat{B}$  - An alternate base point of curve ed25519 in the same order as $B$, chosen as $B$ hashed to a point, proving it's discrete logarithm is unknown.
    \item $l$   - Order of $B$ and $\hat{B}$
    \item $o$   - Order of $G$ and $\hat{G}$
    \item $n$   - Number of bits proved
    \item HashG() - A hash function which outputs an integer $\in \mathbb{Z}_{o}$
    \item HashB() - A hash function which outputs an integer $\in \mathbb{Z}_{l}$
\end{itemize}

\subsection{DLEAGProve($x$)}
\begin{enumerate}
    \item Split $x$ into an array of bits $b$ so $\displaystyle\sum_{i=0}^{n-1}b_{i} = x$
    \item For each $i \in [0, n-2]$ generate random blinding factors:\\
          $r_{i} \in \mathbb{Z}_{o}$ and $s_{i} \in \mathbb{Z}_{l}$
    \item Set the last blinding factors to:\\
          $r_{n-1} = (2^{n-1})^{-1} (-\displaystyle\sum_{i=0}^{n-2} r_{i}2^{i}) \in \mathbb{Z}_{o}$\\
          and\\
          $s_{n-1} = (2^{n-1})^{-1} (-\displaystyle\sum_{i=0}^{n-2} s_{i}2^{i})\in \mathbb{Z}_{l}$\\
          Causing each set of weighted blinding factors to sum to $0$\\
          $\sum_{i=0}^{n-1} r_{i}2^{i} == 0$ and $\sum_{i=0}^{n-1} s_{i}2^{i} == 0$
    \item For each $i \in [0, n-1]$ compute two commitments:
        $C^{G}_{i} = b_{i}G + r_{i}\hat{G} \in \mathbb{Z}_{o}$\\
        $C^{B}_{i} = b_{i}B + s_{i}\hat{B} \in \mathbb{Z}_{l}$\\
        so the weighted commitment sums equal $xG$ and $xB$\\
        $\sum_{i=0}^{n-1} C^{G}_{i}2^{i} = xG$\\
        $\sum_{i=0}^{n-1} C^{B}_{i}2^{i} = xB$
    \item Construct the ring stack, for each $i \in [0, n-1]$:
        \begin{enumerate}
            \item Choose random $j_{i} \in \mathbb{Z}_{o} $ and $ k_{i} \in \mathbb{Z}_{l}$
            \item Set $J_{i,b_{i}} = j_{i}\hat{G}$ and $K_{i,b_{i}} = j_{i}\hat{B}$
            \item For each $z \in [b_{i}+1, 2]$:
                \begin{enumerate}
                    \item $e_{G} = $HashG($C^{G}_{i}||C^{B}_{i}||J||K||i||z$)\\
                          $e_{B} = $HashB($C^{G}_{i}||C^{B}_{i}||J||K||i||z$)
                    \item Choose random $a^{G}_{i,z} \in \mathbb{Z}_{o} $ and $ a^{B}_{i,z} \in \mathbb{Z}_{l}$
                    \item $J_{i,z} = a^{G}_{i,z}\hat{G} - e_{G}(C^{G}_{i} - zG)$\\
                          $K_{i,z} = a^{B}_{i,z}\hat{B} - e_{B}(C^{B}_{i} - zB)$
                \end{enumerate}

        \end{enumerate}
    \item $e^{G}_{0} = HashG(J_{0,2}||...||J_{i-1,2})$ and\\
        $e^{B}_{0} = HashG(K_{0,2}||...||K_{i-1,2})$
    \item Sign the ring stack, For each $i \in [0, n-1]$:
        \begin{enumerate}
            \item $e_{G} = $HashG($C^{G}_{i}||C^{B}_{i}||e^{G}_{0}||e^{B}_{0}||i||0$)
            \item $e_{B} = $HashB($C^{G}_{i}||C^{B}_{i}||e^{G}_{0}||e^{B}_{0}||i||0$)
            \item For each $z \in [0, b_{i}]$:
                \begin{enumerate}
                    \item Choose random $a^{G}_{i,z} \in \mathbb{Z}_{o} $ and $ a^{B}_{i,z} \in \mathbb{Z}_{l}$
                    \item $J_{i,z} = a^{G}_{i,z}\hat{G} - e_{G}(C^{G}_{i} - zG)$\\
                          $K_{i,z} = a^{B}_{i,z}\hat{B} - e_{B}(C^{B}_{i} - zB)$
                    \item $e_{G} = $HashG($C^{G}_{i}||C^{B}_{i}||J||K||i||z+1$)\\
                          $e_{B} = $HashB($C^{G}_{i}||C^{B}_{i}||J||K||i||z+1$)
                \end{enumerate}
            \item Close the loop:\\
                $a^{G}_{i,b_{i}} = j_{i} + e_{G}r_{i}$\\
                $a^{B}_{i,b_{i}} = k_{i} + e_{B}s_{i}$
        \end{enumerate}
    \item Return the proof $(xG, xB, \{C^{G}_{i}\}, \{C^{B}_{i}\}, e^{G}_{0}, e^{B}_{0}, \{a^{G}_{0, i}\}, \{a^{G}_{1, i}\}, \{a^{B}_{0, i}\}, \{a^{B}_{1, i}\})$\\
        Proof length is $33+32+32+32+(33+32+(32)4)n == 129+193n$ bytes.
\end{enumerate}


\subsection{DLEAGVerify(proof)}
$(xG, xB, \{C^{G}_{i}\}, \{C^{B}_{i}\}, e^{G}_{0}, e^{B}_{0}, \{a^{G}_{0, i}\}, \{a^{G}_{1, i}\}, \{a^{B}_{0, i}\}, \{a^{B}_{1, i}\}) = $proof
\begin{enumerate}
    \item Verify the weighted commitment sums equal $xG$ and $xB$\\
        $\sum_{i=0}^{n-1} C^{G}_{i}2^{i} = xG$\\
        $\sum_{i=0}^{n-1} C^{B}_{i}2^{i} = xB$
    \item Verify the ring stack, for each $i \in [0, n-1]$:
        \begin{enumerate}
            \item $J_{i,z} = a^{G}_{i,z}\hat{G} - e_{G}(C^{G}_{i} - zG)$\\
                  $K_{i,z} = a^{B}_{i,z}\hat{B} - e_{B}(C^{B}_{i} - zB)$
            \item $e_{G} = $HashG($C^{G}_{i}||C^{B}_{i}||J||K||i||z+1$)\\
                  $e_{B} = $HashB($C^{G}_{i}||C^{B}_{i}||J||K||i||z+1$)
        \end{enumerate}
    \item $e^{'G}_{0} = HashG(J_{0,2}||...||J_{i-1,2})$ and\\
        $e^{'B}_{0} = HashG(K_{0,2}||...||K_{i-1,2})$
    \item return 1 if $e^{'G}_{0} == e^{G}_{0}$ and $e^{'B}_{0} == e^{B}_{0}$
\end{enumerate}


\section{Non-Interactive Zero Knowledge proof of Discrete Logarithm Equality}
The One-Time VES scheme requires a NIZK DLEQ from \cite{dleq-proof}\\
\\
Given two generator points and two points on the same curve a DLEQ proves the discrete logarithm of the two points is equal.

\subsection{DLEQProve}
Take points $B_{1}, B_{2}$ on the same curve and of the same group order and a scalar $x \in \mathbb{Z}_{o}$ where $o$ is the group order.\\
Hash is a function returning a number $\in \mathbb{Z}_{o}$\\
\begin{enumerate}
    \item Calculate $P_{1} = xB_{1}$ and $P_{2} = xB_{2}$
    \item Generate a random scalar $k \in \mathbb{Z}_{o}$
    \item Calculate $K_{1} = kB_{1}$ and $K2 = kB_{2}$
    \item Set $c = Hash(P_{1} || P_{2} || K_{1} || K2)$
    \item Set $r = k - cx$
    \item The proof is the tuple $(K_{1}, K_{2}, r)$
\end{enumerate}

\subsection{DLEQVerify}
Take points $P_{1}, P_{2}$ and the proof tuple $(K_{1}, K_{2}, r)$ as inputs.
\begin{enumerate}
    \item Set $c = Hash(P_{1} || P_{2} || K_{1} || K_{2}) $
    \item Calculate $R_{1} = rB_{1}$ and $R_{2} = rB_{2}$
    \item Calculate $C_{1} = cP_{1}$ and $C_{2} = cP_{2}$
    \item Output 1 if $K_{1} == R_{1} + C_{1}$ and $K2 == R_{2} + C_{2}$ else 0
\end{enumerate}



\section{One-Time Verifiably Encrypted Signatures}
Also known as Adaptor Signatures.\\
A One-Time VES is a signature made invalid by mixing it with the public key of an encrypting key pair, a valid signature can be decrypted with knowledge of the private encrypting key and the private encrypting key can be recovered with knowledge of both the encrypted and plaintext signatures.
We use a VES constructed to function with ECDSA signatures as described by Fournier et al. in \cite{oneTimeVES}\\

\subsection{EncSign($p_{S}, P_{E}, m$)}
On input of a secret signing key $p_{S}$, a public encryption key $P_{E}$ , and a message $m$, output a ciphertext $\hat{\sigma}$ which is an encrypted signature of $m$ by $p_{S}$.
\begin{enumerate}
    \item Generate a random scalar $r \in \mathbb{Z}_{o}$
    \item Calculate $R_{1} = rG$ and $ R_{2} = rP_{E}$
    \item $\pi = $ DLEQProve($G$, $P_{E}$, $r$)
    \item Set $R_{2x}$ to the $x$ coord of $R_{2} \bmod o$
    \item Calculate $\hat{s} = r^{-1}(Hash(m) + R_{2x}p_{S})$
    \item Return the tuple $(R_{1}, R_{2}, \hat{s}, \pi)$ as $\hat{\sigma}$
\end{enumerate}

\subsection{EncVrfy($P_{S}, P_{E},  m, \hat{\sigma}$)}
On input of a public signing key $P_{S}$, a public encryption key $P_{E}$ , a message $m$ and a ciphertext $\hat{\sigma}$ output
1 if $\hat{\sigma}$ is a valid encryption of a signature on $m$ for $P_{S}$ under $P_{E}$.
\begin{enumerate}
    \item Set $(R_{1}, R_{2}, \hat{s}, \pi) = \hat{\sigma}$
    \item Fail if DLEQVerify($R_{1}, R_{2}, \pi$) != 1
    \item Set $R_{2x} $ to the $x$ coord of $R_{2} \bmod o$
    \item Output 1 if $R_{1} = (Hash(m)G * R_{2x}P_{S})\hat{s}^{-1}$ else 0
\end{enumerate}

\subsection{DecSig($p_{E}, \hat{\sigma}$)}
Decrypt the signature with the private encryption key $p_{E}$.
\begin{enumerate}
    \item Set $(R_{1}, R_{2}, \hat{s}, \pi) = \hat{\sigma}$
    \item Calculate $s = \hat{s}p_{E}^{-1}$
    \item Set $R_{2x}$ to the $x$ coord of $R_{2} \bmod o$
    \item Return the tuple ($R_{2x}, s$) as $\sigma$
\end{enumerate}

\subsection{RecEncKey($P_{E}, \hat{\sigma}$, $\sigma$)}
Recover the private encryption key $p_{e}$ from the plaintext and encrypted signatures.
\begin{enumerate}
    \item Set $(R_{1}, R_{2}, \hat{s}, \pi) = \hat{\sigma}$
    \item Set $(R_{2x}, s) = \sigma$
    \item $p = s^{-1}\hat{s}$
    \item Return p if $pG == P_{E}$ else $-p$
\end{enumerate}


\section{Atomic Swap Protocol}

\subsection{Basic Idea}
Two parties want to exchange coins between blockchains in a such a way
that neither can be cheated.\\
\\
The swap leader exchanges coin from the script-enabled blockchain and sends the first transaction.\\
The leader wants to exchange an amount of coinA for an amount of coinB.\\
The follower wants to exchange an amount of coinB for an amount of coinA.\\


\begin{enumerate}
    \item Leader and follower exchange details necessary for the swap.
    \item Once they agree on the details the leader publishes a lock transaction
which locks up the amount of coinA being exchanged by sending it to an
output that can only be spent if both sides cooperate until the time
agreed for the exchange expires when the leader can reclaim it.
    \item The follower waits for the coinA lock transaction to be confirmed in
it's blockchain.  If he's satisfied the transaction was formed correctly
he publishes a transaction locking up the amount of coinB being
exchanged.  The coinB lock tx can only be spent with knowledge of a
private key from both the leader and follower.
    \item The leader waits for the coinB lock transaction to be confirmed in
it's blockchain. If he's satisfied the transaction was formed correctly
he reveals information to the follower which allows him to spend from
the coinA lock transaction.
    \item By spending from the coinA lock transaction the follower reveals
information to the leader allowing him to spend from the coinB lock
transaction.
    \item The leader spends from the coinB lock transaction and the exchange is
completed successfully.
\end{enumerate}

\noindent Leader:\\
 - Loses: coinA locked value, 1x coinA tx fees and 1x coinB tx fees.\\
 - Gains: coinB locked value.\\
Follower:\\
 - Loses: coinB locked value, 1x coinB tx fees and 1x coinA tx fees.\\
 - Gains: coinA locked value.\\
\\

\subsection{Details}

Step 2:\\
In addition to the coinA lock tx the parties craft a transaction
spending from the coinA lock tx refund path and spendable either by
both parties cooperating or by the follower after a second locktime has
expired.  Before the leader publishes the coinA lock tx he will know the
information to spend the coinA refund tx immediately after it's published
however spending from the coinA refund tx cooperative path reveals
information to the follower allowing him to from the coinB lock
transaction.\\
The refund path could be built into the coinA lock tx, but would
increase the size of the coinA lock tx significantly and should seldom
be required.\\
\\
Step 3:\\
The coinB lock tx outputs to a key which is the sum of a key from the
leader and a key from the follower.\\
The One-Time Verifiably Encrypted Signatures are crafted with this key
as the encryption key, and the information distributed so the
follower reveals his key to the leader by spending from the coinA lock
tx and the leader reveals his key to the follower by spending from the
coinA lock refund tx.\\
\\
\\

If something goes wrong and the leader or follower stop responding:\\
\\
After step 2, before step 3:\\
If the follower drops out:\\
The leader can retrieve his coin from the coinA lock tx by waiting for
the exchange time to expire and publishing the coinA refund tx.  Then
spending from the coinA refund tx cooperative path.  The follower has
not yet locked any coinB.\\
Leader loses 2x coinA tx fees, follower loses nothing.\\
\\
If the leader drops out:\\
The follower can wait for the first locktime to expire, then publish the
coinA refund tx.  Once the coinA refund tx locktime expires the follower
can spend from it with only his signature, thus claiming the amount of
coinA while still retaining the coinB.\\
Leader:\\
 - Loses: 1x coinA fees and locked coinA value.\\
Follower:\\
 - Loses: 1x coinA fees, 1x coinB fees.\\
 - Gains: Locked coinA value\\
\\
\\
After step 3:\\
If the follower drops out:\\
As at step 2, by spending from the coinA refund tx cooperative path the
leader reveals information allowing the follower to spend from the coinB
lock tx.\\
Leader:\\
 - Loses: 2x coinA tx fees.\\
Follower:\\
 - Loses: 2x coinB tx fees.\\
\\
If the leader drops out:\\
As at step 2 except the follower has published the coinB lock tx.\\
Neither party can spend the coinB lock tx without assistance from the
other.\\
Leader:\\
 - Loses: 1x coinA fees and locked coinA value.\\
Follower:\\
 - Loses: 1x coinA fees, 1x coinB fees and locked coinB value.\\
 - Gains: locked coinA value.\\



\subsubsection{Transaction scripts}

\begin{lstlisting}[language=C++, caption={\noindent Lock tx output script}]
OP_IF
    OP_SHA256 {secret_hash} OP_EQUALVERIFY
    OP_2 {pk_leader} {pk_follower} OP_2 OP_CHECKMULTISIG
OP_ELSE
    {lock_for_1} OP_CHECKSEQUENCEVERIFY OP_DROP
    OP_2 {pk_leader_refund} {pk_follower_refund} OP_2 OP_CHECKMULTISIG
OP_ENDIF
\end{lstlisting}

\begin{lstlisting}[language=C++, caption={\noindent Lock refund tx output script}]
OP_IF
    OP_2 {pk_leader_refund} {pk_follower_refund} OP_2 OP_CHECKMULTISIG
OP_ELSE
    {lock_for_2} OP_CHECKSEQUENCEVERIFY OP_DROP
    {pk_follower} OP_CHECKSIG
OP_ENDIF
\end{lstlisting}

\subsubsection{Checking the secret length}
Checking the secret length as required in Decred style atomic swaps
\cite{advisory1} is not strictly necessary as the secret value and hash
will only be used on one blockchain, it may help to prevent unexpectedly
large secret values from being used to throw off the transaction fee.

\section{Future Enhancements}
\begin{itemize}
    \item coinA lock and refund scripts can be joined into one more efficiently on
          chains where the taproot bip-0341\cite{bip-0341} improvement is active.
    \item DLEAG proof could prove multiple bits per ring layer, reducing the proof size.
    \item Improve encoding of VES data to reduce size.
\end{itemize}

\clearpage

\printbibliography

\end{document}
